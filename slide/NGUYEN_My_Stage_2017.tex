\documentclass[11pt]{beamer}
\usetheme{Goettingen}
\usepackage[utf8]{inputenc}
\usepackage[frenchb]{babel}
\usepackage[english]{babel}
\usepackage{amsmath}
\usepackage{amsfonts}
\usepackage{amssymb}
\usepackage{color}




\title[STAGE DE M1 2017]{\textbf{Appliquer la méthode de Newton-Raphson pour résoudre le problème non-linéaire avec Feel++}}  
\author{NGUYEN Thi Tra My} 
\institute[CSMI-M1]{Stage de M1}
\date{28 août 2017} 


%%%%%%%%%%%%%%%%%%%%%%%%%%%%%%%%%%
%%%%%%%%%%%%%%%%%%%%%%%%%%%%%%%%%%
\begin{document}

\begin{frame}
\titlepage
\end{frame}
\begin{frame}
\frametitle{}
\tableofcontents
\end{frame}



%%%%%%%%%%%%%%%%%%%%%%%%%%%%%%%%%%%%%%%%%%%%%%%%%%%%%%%%%%%%%

\section{INTRODUCTION}
\subsection{Objective du stage}

%%%%%%%%%%%%%%%%%%%%%%

\begin{frame}{Objective du stage}
\begin{itemize}
\item Etudier une méthode numérique pour résoudre les problèmes non-linéaires de différents types.\\


\item Documenter au manuel de Feel++\\


\item Vérifier le modèle thermo-electrique en prenent en compt l’effet Seebeck et Peltier sur un semi-conducteur.\\


\end{itemize}
\end{frame}

%%%%%%%%%%%%%%%%%%%%%%



\subsection{Statégie de l'étude}
%%%%%%%%%%%%%%%%%%%%%%

\begin{frame}{Statégie de l'étude}

\begin{itemize}
\begin{figure}
\includegraphics[scale=0.35]{image/i1.png}
\end{figure}
 

\end{itemize}
\end{frame}

%%%%%%%%%%%%%%%%%%%%%%



%%%%%%%%%%%%%%%%%%%%%%%%%%%%%%%%%%%%%%%%%%%%%%%%%%%%%%%%%%%%%

\section{MÉTHODE DE NEWTON-RAPHSON}
\subsection{Unidimention}

%%%%%%%%%%%%%%%%%%%%%%

\begin{frame}{Méthode de Newton-Raphson}
\begin{block}{Fonction d'une variable réelle}
$f(x) = 0$
 \end{block}
 
\begin{figure}
\includegraphics[scale=0.35]{image/n1.png}
\end{figure}


\fcolorbox{red}{white}{$x_{n + 1} = x_n - \frac{f(x_n)}{f'(x_n)}$}

 
 
\end{frame}

%%%%%%%%%%%%%%%%%%%%%%


\subsection{Matrice problème}

%%%%%%%%%%%%%%%%%%%%%%

\begin{frame}{Méthode de Newton-Raphson}
\begin{block}{Systèmes d'équations à plusieurs variables}
$F(X) = 0$
\\
Où
\\
$X =
\begin{bmatrix}
x_1
\\
x_2
\\
\vdots
\\
x_n
\end{bmatrix} ,
F(X) =
\begin{bmatrix}
f_1(x_1, x_2, \cdots, x_n)
\\
f_2(x_1, x_2, \cdots, x_n)
\\
\vdots
\\
f_n(x_1, x_2, \cdots, x_n)
\end{bmatrix} =
\begin{bmatrix}
f_1(x)
\\
f_2(x)
\\
\vdots
\\
f_n(x)
\end{bmatrix}$
 \end{block}
 
 
\begin{block}{Système linéaire}
$ F'(X^k) (X^{k+1} - X^k) = -F(X^k)$
 
\end{block}
 


\begin{itemize}
\item Jacobian matrice: $J(X^{(k)}) = F'(X^k)$

\item Vecteur résidual: $ R^{(k)} = F(X^k)$

\item $ \delta X^{(k)} = X^{(k+1)} - X^{(k)}  $
\end{itemize}
 
\end{frame}

%%%%%%%%%%%%%%%%%%%%%%


%%%%%%%%%%%%%%%%%%%%%%

\begin{frame}

\begin{figure}
\includegraphics[scale=0.3]{image/i2.png}
\end{figure}

\end{frame}

%%%%%%%%%%%%%%%%%%%%%%



%%%%%%%%%%%%%%%%%%%%%%%%%%%%%%%%%%%%%%%%%%%%%%%%%%%%%%%%%%%%%
\section{BRATU PROBLÈME}
\subsection{Problème avec Dirichlet condition}

%%%%%%%%%%%%%%%%%%%%%%
\begin{frame}{Bratu problème avec Dirichlet condition}

\begin{block}{Bratu problème}
\begin{cases}
- \nabla \cdot (\nabla u ) + \lambda e^{u} = 0 \text{ in } \Omega \quad (1)
\\
u = 0 \text{ on } \partial \Omega
\end{cases}
\end{block}

\pause

\begin{block}{Formulation variationnelle}
\text{Determine u } \in H^1_0(\Omega) \text{ satisfying}
\\
\displaystyle\int_{\Omega} \nabla u \cdot \nabla v + \int_{\Omega} \lambda e^{u} v = 0 \quad \forall v \in H^1_0(\Omega)
\end{block}

\pause



\begin{block}{Problème algébraique}
\text{Determine } u_i \text{ satisfying}
\\
A(\varphi_i, \varphi_j) = \displaystyle\int_{\Omega_h} \sum_{i=1}^N u_i \nabla \varphi_i \cdot \nabla \varphi_j
+ \int_{\Omega_h} \lambda e^{\sum_{i=1}^N u_i \varphi_i} \varphi_j = 0 \quad \forall j = 1, \cdots, N

\end{block}

\end{frame}

%%%%%%%%%%%%%%%%%%%%%%




%%%%%%%%%%%%%%%%%%%%%%
\begin{frame}{Applique Newton-Raphson méthode}


\begin{block}{Système non-linéaire}
\begin{cases}
A_1(u_1, \cdots , u_N) = \displaystyle\int_{\Omega_h} \sum_{i=1}^N u_i \nabla \varphi_i \cdot \nabla \varphi_1
+ \int_{\Omega_h} \lambda e^{\sum_{i=1}^N u_i \varphi_i} \varphi_1 = 0
\\
\vdots
\\
A_N(u_1, \cdots , u_N) = \displaystyle\int_{\Omega_h} \sum_{i=1}^N u_i \nabla \varphi_i \cdot \nabla\varphi_N
+ \int_{\Omega_h} \lambda e^{\sum_{i=1}^N u_i \varphi_i} \varphi_N = 0
\end{cases}
\end{block}


\begin{block}{Jacobien matrice}
J_{lj} (u^{(k)}) = \frac{\partial A_l}{\partial u_j} (u^{(k)}) =
\displaystyle\int_{\Omega_h} \nabla \varphi_l \nabla \varphi_j
+ \int_{\Omega_h} \lambda \varphi_l e^{\sum_{i=1}^N u_i \varphi_i} \varphi_j

\end{block}


\begin{block}{Residule}
A_j(u^{(k)}) = \displaystyle\int_{\Omega_h}  \nabla u^{(k)} \cdot \nabla \varphi_j
+ \int_{\Omega_h} \lambda e^{u^{(k)}} \varphi_j
\end{block}


\begin{block}{Initial guess}
u^{(0)} = 0
\end{block}


\end{frame}

%%%%%%%%%%%%%%%%%%%%%%


%%%%%%%%%%%%%%%%%%%%%%
\begin{frame}{Implémentation}

\begin{figure}
\includegraphics[scale=0.5]{image/c2.png}
\end{figure}

\begin{figure}
\includegraphics[scale=0.5]{image/c1.png}
\end{figure}

\begin{figure}
\includegraphics[scale=0.5]{image/c3.png}
\end{figure}

\end{frame}
%%%%%%%%%%%%%%%%%%%%%%



%%%%%%%%%%%%%%%%%%%%%%
\begin{frame}{Méthode d'imposer Dirichlet condition}

\begin{figure}
\includegraphics[scale=0.5]{image/c4.png}
\end{figure}

\pause

\begin{figure}
\includegraphics[scale=0.5]{image/c5.png}
\end{figure}

\end{frame}
%%%%%%%%%%%%%%%%%%%%%%


%%%%%%%%%%%%%%%%%%%%%%
\begin{frame}{Expérience numérique}`

\begin{block}{Forte méthode}

\begin{verbatim}

0  SNES Function norm 1.533744e-01
\\
1  SNES Function norm 1.526050e-03
\\
2  SNES Function norm 3.503014e-05
\\
3  SNES Function norm 8.021951e-07
\\
4  SNES Function norm 1.833466e-08
\\
5  SNES Function norm 4.188840e-10
\\
[env] Time : 5.442774e+00s

\end{verbatim}
\end{block}


\begin{columns}
\column{0.5\textwidth}
\begin{figure}
\includegraphics[scale=0.15]{image/b3.png}
\end{figure}
\column{0.5\textwidth}
\begin{figure}
\includegraphics[scale=0.15]{image/b4.png}
\end{figure}
\end{columns}

\end{frame}
%%%%%%%%%%%%%%%%%%%%%%


%%%%%%%%%%%%%%%%%%%%%%
\begin{frame}

\begin{block}{Faible méthode}

\begin{verbatim}
0  SNES Function norm 1.565416e-01

1  SNES Function norm 1.526007e-03

2  SNES Function norm 3.502907e-05

3  SNES Function norm 8.021703e-07

4  SNES Function norm 1.833408e-08

5  SNES Function norm 4.188708e-10

[env] Time : 1.443950e+00s
\end{verbatim}
\end{block}

\begin{columns}
\column{0.5\textwidth}
\begin{figure}
\includegraphics[scale=0.2]{image/b5.png}
\end{figure}
\column{0.5\textwidth}
\begin{figure}
\includegraphics[scale=0.2]{image/b6.png}
\end{figure}
\end{columns}

\end{frame}
%%%%%%%%%%%%%%%%%%%%%%



%%%%%%%%%%%%%%%%%%%%%%
\begin{frame}

\begin{columns}
\column{0.5\textwidth}
\begin{figure}
\includegraphics[scale=0.2]{image/b12.png}
\end{figure}
\column{0.5\textwidth}
\begin{figure}
\includegraphics[scale=0.2]{image/b13.png}
\end{figure}
\end{columns}

\end{frame}
%%%%%%%%%%%%%%%%%%%%%%


%%%%%%%%%%%%%%%%%%%%%%
\begin{frame}

\begin{figure}
\includegraphics[scale=0.2]{image/fpp5.png}
\end{figure}

\begin{figure}
\includegraphics[scale=0.2]{image/fpp3.png}
\end{figure}


\end{frame}
%%%%%%%%%%%%%%%%%%%%%%



\subsection{Bratu problème mixte}
%%%%%%%%%%%%%%%%%%%%%%
\begin{frame}{Bratu problème mixte}

\begin{block}{Equation non-linéaire avec les conditions mixte}
\begin{cases}
- \nabla \cdot (\nabla u ) + \lambda e^{u} = f \text{ in } \Omega \quad (2)
\\
u = q \text{ on } \partial \Omega_D \text{ (non homogene Dirichlet boundary)}
\\
\nabla u \cdot n = g \text{ on } \partial \Omega_N \text{ (Neumann boundary condition)}
\end{cases}

\end{block}


\pause



\begin{block}{Change variable}
u = u_q + \phi , \quad \phi \in \{ w \in H^1(\Omega), w|_{\partial \Omega_D} = 0 \}
\end{block}

\pause


\begin{block}{Equation non-linéaire avec le condition de Dirichlet homogène}
\begin{cases}

\text{Find } u \in H^1_0(\Omega) \text{ such that }
\\
- \nabla \cdot (\nabla u ) + \lambda e^{u} e^{q} = f
\\
u = 0 \text{ on } \partial \Omega_D
\\
\nabla u \cdot n = g \text{ on } \partial \Omega_N

\end{cases}
\end{block}

\end{frame}

%%%%%%%%%%%%%%%%%%%%%%

%%%%%%%%%%%%%%%%%%%%%%
\begin{frame}
\begin{block}{Matrice Jacobien}
J_{lj} (u^{(k)}) = \frac{\partial A_l}{\partial u_j} (u^{(k)}) =
\displaystyle\int_{\Omega_h} \nabla \varphi_l \nabla \varphi_j
+ \int_{\Omega_h} \lambda \varphi_l e^{\sum_{i=1}^N u_i \varphi_i} e^{q} \varphi_j

\end{block}


\begin{block}{Residual}
A_j(u^{(k)}) = \displaystyle\int_{\Omega_h}  \nabla u^{(k)} \cdot \nabla \varphi_j - \int_{\partial \Omega_N} g \varphi_j + \int_{\Omega_h} \lambda e^{u^{(k)}} e^{q} \varphi_j - \int_{\Omega_h} f \varphi_j
\end{block}

\end{frame}

%%%%%%%%%%%%%%%%%%%%%%

%%%%%%%%%%%%%%%%%%%%%%

\begin{frame}
\begin{block}{Matrice Jacobien}
\begin{verbatim}

a +=on(_range=markedfaces(mesh,"Dirichlet"),_rhs = l, _element=u, _expr = cst(0.) );

\end{verbatim}
\end{block}


\begin{block}{Residual}
\begin{verbatim}

w.on(_range=markedfaces(mesh,"Dirichlet"),_expr = cst(0.));
\end{verbatim}

\end{block}


\begin{block}{Initial guess}
\begin{verbatim}

u.on(_range = elements(mesh), _expr = q);
\end{verbatim}
\end{block}


\end{frame}


%%%%%%%%%%%%%%%%%%%%%%

%%%%%%%%%%%%%%%%%%%%%%
\begin{frame}{Exemple 1}
\begin{figure}
\includegraphics[scale=0.2]{image/m1.png}
\end{figure}

\begin{block}{Les conditions}
\begin{verbatim}

g = 0:x:y

f = 0:x:y

q = 0
\end{verbatim}
\end{block}

\end{frame}



%%%%%%%%%%%%%%%%%%%%%%

%%%%%%%%%%%%%%%%%%%%%%
\begin{frame}

\begin{verbatim}

0  SNES Function norm 8.601097e-02

1  SNES Function norm 3.547906e-04

2  SNES Function norm 3.060074e-06

3  SNES Function norm 2.633270e-08

4  SNES Function norm 2.265028e-10

[env] Time : 1.936892e+00s
\end{verbatim}


\begin{columns}
\column{0.5\textwidth}
\begin{figure}
\includegraphics[scale=0.15]{image/m2.png}
\end{figure}
\column{0.5\textwidth}
\begin{figure}
\includegraphics[scale=0.15]{image/m3.png}
\end{figure}
\end{columns}

\end{frame}

%%%%%%%%%%%%%%%%%%%%%%



%%%%%%%%%%%%%%%%%%%%%%
\begin{frame}{Exemple 2}

\begin{verbatim}

g = 0:x:y; 
f = 0:x:y; 
q = 2:x:y
\end{verbatim}


\begin{verbatim}

0  SNES Function norm 4.696040e+00

1  SNES Function norm 2.435605e+00

2  SNES Function norm 4.867167e-01

3  SNES Function norm 4.270286e-02

4  SNES Function norm 1.393261e-02

5  SNES Function norm 4.844162e-03

6  SNES Function norm 1.609185e-03

7  SNES Function norm 5.262403e-04

8  SNES Function norm 1.714626e-04

9  SNES Function norm 5.580652e-05

10  SNES Function norm 1.815940e-05

11  SNES Function norm 5.908626e-06

12  SNES Function norm 1.922498e-06

13  SNES Function norm 6.255228e-07

14  SNES Function norm 2.035262e-07

15  SNES Function norm 6.622123e-08

16  SNES Function norm 2.154638e-08

[env] Time : 3.146386e+00s
\end{verbatim}


\end{frame}


%%%%%%%%%%%%%%%%%%%%%%


\begin{frame}
\begin{columns}
\column{0.5\textwidth}
\begin{figure}
\includegraphics[scale=0.15]{image/m4.png}
\end{figure}
\column{0.5\textwidth}
\begin{figure}
\includegraphics[scale=0.15]{image/m5.png}
\end{figure}
\end{columns}
\end{frame}


%%%%%%%%%%%%%%%%%%%%%%
\begin{frame}{Exemple 3}

\begin{verbatim}

g = 0:x:y; 
f = 0:x:y; 
q=sin(pi*x)*cos(pi*y):x:y
\end{verbatim}


\begin{verbatim}

0  SNES Function norm 9.130287e-01

1  SNES Function norm 2.650988e-02

2  SNES Function norm 4.161965e-03

3  SNES Function norm 6.305129e-04

4  SNES Function norm 9.543921e-05

5  SNES Function norm 1.440296e-05

6  SNES Function norm 2.172523e-06

7  SNES Function norm 3.276372e-07

8  SNES Function norm 4.940857e-08

9  SNES Function norm 7.450841e-09

[env] Time : 2.707165e+00s
\end{verbatim}


\end{frame}

%%%%%%%%%%%%%%%%%%%%%%

%%%%%%%%%%%%%%%%%%%%%%

\begin{frame}
\begin{columns}
\column{0.5\textwidth}
\begin{figure}
\includegraphics[scale=0.15]{image/m13.png}
\end{figure}
\column{0.5\textwidth}
\begin{figure}
\includegraphics[scale=0.15]{image/m14.png}
\end{figure}
\end{columns}
\end{frame}


%%%%%%%%%%%%%%%%%%%%%%

%%%%%%%%%%%%%%%%%%%%%%





%%%%%%%%%%%%%%%%%%%%%%
\begin{frame}{Exemple 4}

\begin{verbatim}

g = x^2 + y^2:x:y; 
f = exp(x^2):x:y; 
q=sin(pi*x)*cos(pi*y):x:y
\end{verbatim}


\begin{verbatim}

0  SNES Function norm 8.234561e-01

1  SNES Function norm 8.242994e-03

2  SNES Function norm 2.013958e-04

3  SNES Function norm 6.408645e-06

4  SNES Function norm 2.411853e-07

5  SNES Function norm 9.791356e-09

6  SNES Function norm 4.093998e-10

[env] Time : 2.011367e+00s
\end{verbatim}


\end{frame}

%%%%%%%%%%%%%%%%%%%%%%

%%%%%%%%%%%%%%%%%%%%%%

\begin{frame}
\begin{columns}
\column{0.5\textwidth}
\begin{figure}
\includegraphics[scale=0.2]{image/m9.png}
\end{figure}
\column{0.5\textwidth}
\begin{figure}
\includegraphics[scale=0.2]{image/m10.png}
\end{figure}
\end{columns}
\end{frame}


%%%%%%%%%%%%%%%%%%%%%%


%%%%%%%%%%%%%%%%%%%%%%
\begin{frame}{Exemple 3D}

\begin{verbatim}

g = x^2 + y^2 +z^2:x:y:z; 
f = 0:x:y:z; 
q=sin(pi*x)*cos(pi*y):x:y:z
\end{verbatim}


\begin{verbatim}

0  SNES Function norm 2.304001e-01

1  SNES Function norm 5.415787e-03

2  SNES Function norm 4.168578e-04

3  SNES Function norm 6.551548e-05

4  SNES Function norm 1.237700e-05

5  SNES Function norm 2.399950e-06

6  SNES Function norm 4.678230e-07

7  SNES Function norm 9.129960e-08

8  SNES Function norm 1.782352e-08

9  SNES Function norm 3.479791e-09

10  SNES Function norm 6.793947e-10

[env] Time : 7.274972e+01s
\end{verbatim}

\end{frame}

%%%%%%%%%%%%%%%%%%%%%%

\begin{frame}
\begin{columns}
\column{0.5\textwidth}
\begin{figure}
\includegraphics[scale=0.15]{image/p13.png}
\end{figure}
\column{0.5\textwidth}
\begin{figure}
\includegraphics[scale=0.15]{image/p16.png}
\end{figure}
\end{columns}

\column{0.5\textwidth}
\begin{figure}
\includegraphics[scale=0.15]{image/p17.png}
\end{figure}
\column{0.5\textwidth}
\begin{figure}
\includegraphics[scale=0.15]{image/p18.png}
\end{figure}
\end{columns}
\end{frame}


%%%%%%%%%%%%%%%%%%%%%%

\begin{frame}{Le cas le solveur ne converge pas}
\begin{verbatim}

a +=on(_range=markedfaces(mesh,"Dirichlet"),_rhs = l, _element=u, _expr = q

\end{verbatim}


\begin{verbatim}

 48  SNES Function norm 7.897080e-06
 
 49  SNES Function norm 6.676722e-06
 
 50  SNES Function norm 6.599377e-06
 
E0824 01:44:15.905316  1066 solvernonlinearpetsc.cpp:989] Nonlinear solve did not converge due to DIVERGED_MAX_IT iterations 50

E0824 01:44:15.905508  1066 backend.cpp:464]
[backend] non-linear solver fail

E0824 01:44:15.906250  1066 backend.cpp:465] Backend  : non-linear solver failed to converge


\end{verbatim}

\end{frame}


%%%%%%%%%%%%%%%%%%%%%%

\section{BRATU P-LAPLACIEN}

%%%%%%%%%%%%%%%%%%%%%%%%%%%%%%%%%%%%%%%%%%%%%%%%%%%%%
\begin{frame}{Problème Bratu p-laplacien}
\begin{cases}
- \nabla \cdot (\eta \nabla u ) + \lambda e^{u} = 0 \text{ in } \Omega \quad (1)
\\
\text{with } \eta = \left( \epsilon^2 + \gamma  \right)^{\frac{p-2}{2}}
\text{ and } \eta = 1/2 | \nabla u|^2
\\
u = 0 \text{ on } \partial \Omega
\end{cases}

\pause

\begin{block}{ALgébraique problème}
A(\varphi_i, \varphi_j) = \displaystyle\int_{\Omega_h} \left( \epsilon^2 + 1/2 \left| \sum_{i=1}^N u_i \nabla \varphi_i \right|^2 \right)^{\frac{p-2}{2}} \sum_{i=1}^N u_i \nabla \varphi_i \cdot \nabla \varphi_j
+ \int_{\Omega_h} \lambda e^{\sum_{i=1}^N u_i \varphi_i} \varphi_j = 0 \quad \forall j = 1, \cdots, N

\end{block}
\end{frame}



\begin{frame}{3D avec p = 2.1, h=0.1}

\begin{verbatim}

0  SNES Function norm 1.346444e-02

1  SNES Function norm 5.568333e-03

2  SNES Function norm 1.985630e-03

3  SNES Function norm 6.474841e-04

...

14  SNES Function norm 1.960911e-10

15  SNES Function norm 3.608911e-11

[env] Time : 9.216519e+01s

\end{verbatim}
\end{frame}



\begin{frame}

\begin{columns}
\column{0.5\textwidth}
\begin{figure}
\includegraphics[scale=0.15]{image/pl3.png}
\end{figure}
\column{0.5\textwidth}
\begin{figure}
\includegraphics[scale=0.15]{image/pl8.png}
\end{figure}
\end{columns}

\column{0.5\textwidth}
\begin{figure}
\includegraphics[scale=0.15]{image/pl4.png}
\end{figure}
\column{0.5\textwidth}
\begin{figure}
\includegraphics[scale=0.15]{image/pl5.png}
\end{figure}
\end{columns}

%\begin{figure}
%\includegraphics[scale=0.15]{image/pl10.png}
%\end{figure}

\end{frame}



\begin{frame}{Case 3D avec p = 2.5, h=0.1}
\begin{verbatim}

48  SNES Function norm 7.897080e-06

49  SNES Function norm 6.676722e-06

50  SNES Function norm 6.599377e-06

E0824 01:44:15.905316  1066 solvernonlinearpetsc.cpp:989] Nonlinear solve did not converge due to DIVERGED_MAX_IT iterations 50

E0824 01:44:15.905508  1066 backend.cpp:464]
[backend] non-linear solver fail

E0824 01:44:15.906250  1066 backend.cpp:465] Backend  : non-linear solver failed to converge

\end{verbatim}
\end{frame}



\begin{frame}

\begin{figure}
\includegraphics[scale=0.4]{image/i5.png}
\end{figure}
\end{frame}




\begin{frame}

\begin{columns}
\column{0.5\textwidth}
\begin{figure}
\includegraphics[scale=0.15]{image/pl11.png}
\end{figure}
\column{0.5\textwidth}
\begin{figure}
\includegraphics[scale=0.15]{image/pl12.png}
\end{figure}
\end{columns}

\end{frame}


\begin{frame}{Première observation}

\begin{itemize}
\item Le temps calcule est raisonnable.

\item pc-type=gasm and pc-type=lu donne la même résultat.

\item Le résultat est fiable.

\item Le calcule coût cher quand hsize est petit et les conditions sont compliqués

\end{itemize}

\end{frame}


\section{MODÈLE DE L'EFFET PELTIER-SEEBECK}


\begin{frame}{Module de l'effet Peltier-Seebeck}
\begin{figure}
\includegraphics[scale=0.3]{image/s1.png}
\end{figure}
\end{frame}

\begin{frame}{Module de l'effet Peltier-Seebeck}

\begin{block}{La conservation de l'énergie}
\nabla \cdot q = Q
\end{block}

\begin{block}{La conservation de la charge électrique}
\nabla \cdot j = 0
\end{block}

\begin{block}{L'équation de potentiel électrique}
E = - \nabla V
\end{block}

\begin{block}{L'équation de tranfert de chaleur}
q = -k \nabla T + Pj
\\
j = \sigma(E - \alpha \nabla T)
\end{block}

\begin{block}{L'effet de Joule}
Q = j \cdot E
\end{block}


\begin{block}{L'effet de JPeltier-Seebeck}
P = \alpha T
\end{block}

\end{frame}



\begin{frame}{Les conditions aux limits}
+ Thermic

Ground : $T = T_g = 0°C$ (Dirichlet)

Adiabatic: $\nabla T \cdot n = 0$ (Neumann)

Intensity: $\nabla T \cdot n = 0$ (Neumann)

+ Electric

Ground : $V = V_g = 0 V$ (Dirichlet)

Adiabatic: $\nabla V \cdot n = 0$ (Neumann)

Intensity: $\nabla V \cdot n = V_N = 0.7 A$ (Neumann)

* Adjoint condition in $\Omega_i \cap \Omega_j$

$q \cdot n_i = - q \cdot n_j$

$j \cdot n_i = - j \cdot n_j$

$T_i = T_j$

$V_i = V_j$
\end{frame}



\begin{frame}

\begin{block}{Formulation forte du problème}
- \nabla \cdot \left( (\sigma \alpha^2 T + k) \nabla T + \alpha \sigma T \nabla V \right)
= \sigma |\nabla V|^2 + \sigma \alpha \nabla T \cdot \nabla V
\\
- \nabla \cdot (\sigma \alpha \nabla T + \sigma \nabla V) = 0
\end{block}


\begin{block}{Forme matricielle}
- \nabla \cdot
\left(
\begin{bmatrix}
\sigma \alpha^2 T + k  & \sigma \alpha T
\\
\sigma \alpha & \sigma

\end{bmatrix}
\begin{bmatrix}
\nabla T
\\
\nabla V
\end{bmatrix}
\right)
=
\begin{bmatrix}
\sigma |\nabla V|^2 + \sigma \alpha \nabla T \cdot \nabla V
\\
0
\end{bmatrix}
\end{block}


\begin{block}{Formuleation variationelle}
\text{Find } \left( T , V\right) \in H^1_{\Gamma_D,T_g}(\Omega) \times H^1_{\Gamma_D,V_g}(\Omega)
\text{ for all } \left( t , v \right) \in H^1_{\Gamma_D,0}(\Omega) \times H^1_{\Gamma_D,0}(\Omega)
\text{ such that}
\\
\sum^{N_{mat}}_{i = 1} \left( - \displaystyle\int_{\Omega_i} q \cdot \nabla t
\right)
= \sum^{N_{mat}}_{i = 1} \left( \displaystyle\int_{\Omega_i} Q t \right)
\\
\sum^{N_{mat}}_{i = 1} \left( - \displaystyle\int_{\Omega_i} j \cdot \nabla v
\right)
= - \int_{Intensity} V_N v
\end{block}

\end{frame}


\begin{frame}{Implementation}
\begin{block}{Definire l'espace de fonction}

\begin{verbatim}

typedef Mesh< Simplex< FEELPP_DIM,1 > > mesh_type;

auto mesh = loadMesh(_mesh=new mesh_type);

typedef FunctionSpace<mesh_type, bases<Lagrange<1,Scalar>,Lagrange<1,Scalar> > > space_type;


\\


auto Vh = space_type::New(_mesh=mesh);
    
auto TV = Vh->element();

auto T = TV.element<0>();

auto V = TV.element<1>();
\end{verbatim}
\end{block}


\begin{figure}
\includegraphics[scale=0.6]{image/i6.png}
\end{figure}

\end{frame}


\begin{frame}
\begin{columns}
\column{0.5\textwidth}
\begin{figure}
\includegraphics[scale=0.2]{image/s2.png}
\end{figure}
\column{0.5\textwidth}
\begin{figure}
\includegraphics[scale=0.2]{image/s3.png}
\end{figure}
\end{columns}
\end{frame}


\begin{frame}
\begin{columns}
\column{0.5\textwidth}
\begin{figure}
\includegraphics[scale=0.2]{image/s4.png}
\end{figure}
\column{0.5\textwidth}
\begin{figure}
\includegraphics[scale=0.2]{image/s5.png}
\end{figure}
\end{columns}
\end{frame}


\begin{frame}
\begin{figure}
\includegraphics[scale=0.2]{image/s61.png}
\end{figure}

\begin{verbatim}

0  SNES Function norm 1.219878e-01

1  SNES Function norm 6.870446e-03

2  SNES Function norm 3.242836e-06

3  SNES Function norm 5.515413e-11

T_electrode1 = 2.285343e+02

T_electrode2 = 2.731662e+02`

T_Adiabatic = 2.508503e+02

T_mean = 2.731662e+02

[env] Time : 2.226648e+01s
\end{verbatim}

\end{frame}



\begin{frame}{Rajouter l'option}

\begin{columns}
\column{0.5\textwidth}
\begin{figure}
\includegraphics[scale=0.2]{image/s16.png}
\end{figure}
\column{0.5\textwidth}
\begin{figure}
\includegraphics[scale=0.2]{image/s17.png}
\end{figure}
\end{columns}
\end{frame}


\end{frame}



\begin{frame}

\begin{columns}
\column{0.5\textwidth}
\begin{figure}
\includegraphics[scale=0.2]{image/i11.png}
\end{figure}
\column{0.5\textwidth}
\begin{figure}
\includegraphics[scale=0.2]{image/i12.png}
\end{figure}
\end{columns}
\end{frame}


\end{frame}




\section{CONCLUSION}

\begin{frame}{Conclusion}

\begin{itemize}

\item La méthode est 'délicate', elle nécessite une grand précision dans le cacule
pour qu'elle soit converge.

\item Le choix d'initial guess est très important. Il demande que l'initial guess
soit proche du résolution. Il y a des techniques spécifique pour le traiter comme
l'algorithme de continuation

\item Le grand erreur dans la première estimate peut conduire à la non convergence
de l'algorithme.

\item En générale, la convergence est quadratique.

\item Le calcule coût cher quand le maillage est fin et les conditions sont compliqué.

\item Les fonctions de faire mettre à jour le matrice Jacobien et le vecteur résidul
demandent de faire attention sur les références des variables.

\end{itemize}
\end{frame}


\begin{frame}{Perspective}

\begin{itemize}

\item Analyser plus profondement la vistess
de convergent de la méthode dans plusieurs cases et de comparer avec les autres
méthodes.

\item on pourrait utiliser GINAC pour faciliter
les calcules dérivées et pour construire un programme plus générale pour plusieurs
modèle.

\end{itemize}
\end{frame}



\begin{frame}  
\begin{center}
\large MERCI DE VOTRE ATTENTION
\end{center}
\end{frame} 

\end{document}

 

%%0:11:27 2/3/2014Last Modification of contents
%%0:14:32 2/3/2014Last Modification of contents
